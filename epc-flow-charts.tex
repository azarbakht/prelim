% Model flow charts
% Author: Fabian Schuh
\documentclass{minimal}

\usepackage{pgf}
\usepackage{tikz}
\usepackage{tcolorbox}
%%%<
\usepackage{verbatim}
\usepackage[active,tightpage]{preview}
\PreviewEnvironment{tikzpicture}
\setlength\PreviewBorder{5pt}%
%%%>

\begin{comment}
:Title:  Model flow charts
:Grid: 1x2


\end{comment}

\usepackage[utf8]{inputenc}
\usetikzlibrary{arrows,automata}
\usetikzlibrary{positioning}
\usetikzlibrary{shapes.geometric, arrows, positioning, calc, matrix}

\tikzset{
    state/.style={
           rectangle,
           rounded corners,
           draw=black, very thick,
           minimum height=2em,
           inner sep=2pt,
           text centered,
           },
}

\begin{document}

\begin{tikzpicture}[scale=1.0, align=center, on grid, auto, ->,>=stealth']

 % First node
 % STATE Data Collection
 % Use previously defined 'state' as layout (see above)
 % use tabular for content to get columns/rows
 % parbox to limit width of the listing
 \node[state] (Data Collection) 
 {\begin{tabular}{l}
  \textbf{Data Collection}\\
  \parbox{4cm}{
  \begin{itemize}
   \item Mailing Lists
   \item Bug Tracking Repositories
   \item Codebase
  \end{itemize}
  }
  \end{tabular}};

 % STATE Data Cleaning and Wrangling
 \node[state,       % layout (defined above)
 node distance=2cm,     % distance to Data Collection
 text width=5.5cm,        % max text width
 yshift = -2cm,
 xshift = 1cm,
 right of=Data Collection,        % Position is to the right of Data Collection
 ] (Data Cleaning and Wrangling)    % move 3cm in y
 {%                     % posistion relative to the center of the 'box'
 \begin{tabular}{l}     % content
  \textbf{Data Cleaning and Wrangling}\\
  \parbox{5cm}{12 equioespaced directed graphs for each project
}
 \end{tabular}
 };

 % STATE Morkov Chain Monte Carlo Estimation
 \node[state,
 right of=Data Cleaning and Wrangling,
 node distance=2cm,
 text width=7cm, 
 yshift=-2cm] (Morkov Chain Monte Carlo Estimation) 
 {%
 \begin{tabular}{l}
  \textbf{Morkov Chain Monte Carlo Estimation}\\
  \parbox{6cm}{
  \begin{itemize}
  \item Rate of Change
  \item Parameter Estimates with p-value and s.e.
  \end{itemize}
  }
 \end{tabular}
 };

 % STATE Model
 \node[state,
 below of=Morkov Chain Monte Carlo Estimation,
 node distance=1cm, yshift=-1cm, xshift = 2cm] (Model) 
 {%
 \begin{tabular}{l}
  \textbf{Statistical Model}\\
  \parbox{5cm}{
	\begin{itemize}
	\item Test of Goodness of Fit
	\item Relative Importance of Effects
\end{itemize}
  }
 \end{tabular}
 };


 % STATE MANOVA
 \node[state,
 below of=Model,
 node distance=1cm, yshift=-1cm, xshift = 3cm] (MANOVA) 
 {%
 \begin{tabular}{l}
  \textbf{Multi-Parameter T-test and MANOVA}\\
  \parbox{6cm}{
	\begin{itemize}
	\item Test of Goodness of Fit
	\item Relative Importance of Effects
\end{itemize}
  }
 \end{tabular}
 };
 
% % STATE Results
% \node[state,       % layout (defined above)
% node distance=5cm,     % distance to Data Collection
% text width=7cm,        % max text width
% below of=MANOVA,
% xshift= 1cm,        
% ] (Results) {
%    \begin{tcolorbox}[%
%        title={Results},
%    ]
%	\begin{itemize}
%	\item Reresented Collaboration with Longitudinal Change
%	\item Modeled change and Rate of change statistically
%	\item Expressed underlying properties/values of community Behavior as model effects and their significance and relative importance
%	\item Could be a starting point for gaining an understanding of longitudinal change and underlying properties of an open source project community
%	\end{itemize}
%    \end{tcolorbox}
%};

 \node[state,
 below of=MANOVA,
 node distance=1cm, yshift=-2.2cm, xshift = 1cm] (Results) 
 {%
 \begin{tabular}{l}
  \textbf{Results}\\
  \parbox{8cm}{
\begin{itemize}
	\item Reresented Collaboration with Longitudinal Change
	\item Modeled change and Rate of change statistically
	\item Expressed underlying properties/values of community Behavior as model effects and their significance and relative importance
	\item Good starting point for gaining an understanding of longitudinal change of underlying properties of an open source project community
	\end{itemize}
  }
 \end{tabular}
 };

 % draw the paths and and print some Text below/above the graph
 \path 
(Data Collection) edge[bend right=-10]
%node[anchor=south,right,text width=1cm]
%{Raw Data}
(Data Cleaning and Wrangling)
                   
(Data Cleaning and Wrangling) 
edge [bend right=-10]
(Morkov Chain Monte Carlo Estimation)

 
(Morkov Chain Monte Carlo Estimation)
edge[bend right=-10]
%node[anchor=east,right,text width=6cm,xshift=1em]
%{3} 
(Model)

(Model)
edge[bend right=-10]
(MANOVA)

(MANOVA)
edge[bend right=-20]
(Results)

% (Morkov Chain Monte Carlo Estimation)  edge[bend left=45] node[anchor=north,below,text width=6cm,yshift=-1em,xshift=-2em]
%                  {X31
%                  } (Data Collection)
 ;
\end{tikzpicture}
\end{document}
